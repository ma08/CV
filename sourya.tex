%% start of file `template.tex'.
%% Copyright 2006-2010 Xavier Danaux (xdanaux@gmail.com).
%% Copyright 2010-2011 Mark Liu (markwayneliu@gmail.com).
%
% This work may be distributed and/or modified under the
% conditions of the LaTeX Project Public License version 1.3c,
% available at http://www.latex-project.org/lppl/.

\documentclass[11pt,a4paper,roman]{moderncv}

\usepackage{verbatim}

% moderncv themes
\moderncvstyle{classic}
\moderncvcolor{blue}

% character encoding
\usepackage[utf8]{inputenc}                   % replace by the encoding you are using

% adjust the page margins
\usepackage[scale=0.85,top=1cm, margin=2cm]{geometry}

%\setlength{\hintscolumnwidth}{3cm}						% if you want to change the width of the column with the dates
%\AtBeginDocument{\setlength{\maketitlenamewidth}{6cm}}  % only for the classic theme, if you want to change the width of your name placeholder (to leave more space for your address details
%\AtBeginDocument{\recomputelengths}                     % required when changes are made to page layout lengths

%\usepackage{hyperref}
%\hypersetup{
    %colorlinks=true,
    %linkcolor=blue,
    %filecolor=magenta,      
    %urlcolor=cyan
%}
%\renewcommand*{\namefont}{\fontsize{42}{40}\mdseries\upshape}
% personal data
\firstname{Sourya}
\familyname{Kakarla}
\title{}           
\address{39-10-11, Veterinary Hospital Street, Labbipet, Vijayawada}{Andhra Pradesh}    % optional, remove the line if not wanted
\mobile{(+919493250901)}                    % optional, remove the line if not wanted
\email{sourya4@gmail.com}                      % optional, remove the line if not wanted
\social[github]{ma08}
%\social[stackoverflow][stackoverflow.com/users/3465519/ma08]{ma08}
\homepage{ma08.github.io}                % optional, remove the line if not wanted
\extrainfo{\httplink[stackoverflow/ma08]{stackoverflow.com/users/3465519/ma08}} % optional, remove the line if not wanted

%\renewcommand\cventry[6]{\cventry{#1}{#2}{#3}{#4}{#5}{\fontsize{12}{15}\selectfont#6 }}
\renewcommand*\titlefont{\fontsize{15}{24}\selectfont}
\newcommand\Mycvline[2]{%
\cvline{#1}{\selectfont \fontfamily{\sfdefault}\fontsize{12}{15}\selectfont#2 }}
% to show numerical labels in the bibliography; only useful if you make citations in your resume
%\makeatletter
%\renewcommand*{\bibliographyitemlabel}{\@biblabel{\arabic{enumiv}}}
%\makeatother

%\nopagenumbers{}                             % uncomment to suppress automatic page numbering for CVs longer than one page
%----------------------------------------------------------------------------------
%            content
%----------------------------------------------------------------------------------
\begin{document}

\vspace*{-2\baselineskip}
\maketitle
\vspace*{-1.5\baselineskip}
%\vspace*{-8mm}
\pagestyle{empty}


\section{Education}
\cventry{\textbf{2020--}}{PhD, Crypto Research Lab, Computer Science}{}{IIT Kharagpur, W.B}{}{}
%\cvline{CGPA:}{\small 8.67/10}
\cventry{\textbf{2012--2017}}{Computer Science and Engineering (B.Tech-M.Tech)}{}{IIT Kharagpur, W.B}{}{}
\cvline{CGPA:}{\small 8.67/10}
\cventry{\textbf{2010--2012}}{Intermediate}{}{Sri Chaitanya Junior College, Hyderabad, Telangana}{}{}
\cvline{Aggregate:}{\small 94.7\%}
\cventry{\textbf{2010}}{Secondary School}{}{V.P.S Public School, Vijayawada, Andhra Pradesh}{}{}  % arguments 3 to 6 can be left empty
\cvline{CGPA:}{\small 10/10 (CBSE)}

\section{Experience}
\cventry{Oct,2019 - Present}{Junior Research Fellow}{}{IIT Kharagpur}{}{
\begin{description}
  \item[$\bullet$] Working on the project Cryptanalysis of Cryptographic Ciphers with Emphasis on AES and RSA. Sponsoring Agency: Meity, Govt. of India
\end{description}
}
\cventry{July,2017 - Sep,2019}{Software Engineer}{}{Microsoft India(R\&D) Pvt. Ltd, Hyderabad, Telangana}{}{
\begin{description}
  \item[$\bullet$] Worked as backend developer.
  \item[$\bullet$] Worked on Azure Service Fabric to develop and maintain different services.
\end{description}
}
\cventry{May,2015 - July,2015}{Intern}{}{Microsoft India(R\&D) Pvt. Ltd, Hyderabad, Telangana}{}{
\begin{description}
  \item[$\bullet$] Worked on building an Android app.
  \item[$\bullet$] Implemented the complete backend on .NET using various services of Microsoft Azure.
\end{description}
}

%\bigskip
\cventry{May,2014 - July,2014}{Intern}{}{Advitech Infotech Pvt. Ltd, Hyderabad, Telangana}{}{
\begin{description}
  \item[$\bullet$] Worked on MEAN Stack to develop a social webapp for professionals.
  \item[$\bullet$] Primarily worked on the server side using  node.js and mongoDB.
  %\item[$\bullet$] Implemented the connections part,  user groups and admin policy for groups, persistent and real time notifications. Worked on both frontend and backend.
\end{description}
}
%\bigskip
\section{Award Winning Projects}
\cventry{2014-2015}{ChatterBox}{Multiple Hackathons by IBM, Flipkart, and Microsoft}{}{}{
\begin{description}
\item[$\bullet$] Built a Windows Store app for Microsoft's Code.Fun.Do 2014 hackathon. Secured \textbf{1st place} in the Kharagpur competition. Secured a position in the \textbf{top 5 (Winners)} teams in the final stage of the competition which was conducted nationwide.
\item[$\bullet$] Built a web application using node.js, mongoDB and Angular for GES Hackathon organized by IBM\@. Secured \textbf{2nd place} in the hackathon.
\item[$\bullet$] Built a topic recognition utility over chat using reddit data. Secured \textbf{winning position} in hackathon organized by Flipkart in 2015.
\item[$\bullet$] Built a Chrome extension for facebook chat for Microsoft's Code.Fun.Do 2015 hackathon. Secured \textbf{1st place} in the Kharagpur competition. 
\item[$\bullet$] A chat organizing application which makes navigating/accessing chat history easier by indexing and providing various other utilities. Developed by a team of 4.
\end{description}
}

\section{Academic Projects}
\cventry{Mar,2015-May,2017}{Thesis Project - Cryptanalysis of AES and AES based ciphers}{}{}{}{
\begin{description}
\item[$\bullet$] Studied and implemented 5-round attack on AES-128 using Impossible Differential cryptanalysis.
\item[$\bullet$] Extended 5-round attack to AES-256 and a partial attack on 6-round AES-256.
\item[$\bullet$] Developed and implemented methods to reduce the RAM usage from 4 TB to 176 GB.
\item[$\bullet$] Proposed and implemented meet-in-the-middle attack on reduced round PAEQ, a 2nd round candidate in the CAESAR competition for Authenticated Encryption schemes.
\end{description}
}

\cventry{Aug 15}{Normalizing Chat \& Identifying Topic of Conversation}{Natural Language Processing}{}{}{
\begin{description}
  \item[$\bullet$] This project aims to identify the topic of conversation in Social Media, especially Chat.
  %\item[$\bullet$] It involves normalizing the chat text to a standard format and then classifying the chat conversation to a particular topic.
  \item[$\bullet$] Worked on heuristic algorithms for predicting the topic of conversation.
\end{description}
}


%\cventry{2014}{Automated Restoration of Torn Documents}{}{Opensoft, IIT Kharagpur}{}{
%\begin{description}
%\item[$\bullet$] Team(of 7) secured \textbf{bronze medal} in the competition.
%\item[$\bullet$] Used openCV on Python for image processing.
%\item[$\bullet$] Was primarily responsible for the GUI - implemented using PyQt - a python wrapper for Qt.
%\item[$\bullet$] Implemented drag and drop functionality to upload and remove images, implemented animation effects for loading - using PyQt.
%\item[$\bullet$] Backend -  Implemented a function to calculate Overlapping Area of two polygons using Polygon module.
%\item[$\bullet$] Github: \httplink{github.com/srinivasans/AutoMosaic}. 
%\end{description}
%}

%\cventry{2015}{PlotIt}{}{Opensoft, IIT Kharagpur}{}{
%\begin{description}
%\item[$\bullet$] Team(of 10) secured \textbf{4th place} in the competition. 
%\item[$\bullet$] Developed in python. Used sympy, matplotlib and mayavi for plotting.
%\item[$\bullet$] Was primarily responsible for integrating sympy and mayavi's plotting interface with the GUI built using PyQt - a python wrapper for Qt.
%\item[$\bullet$] Implemented plotting explicit, implicit and parametric equations in 2D and 3D.
%\end{description}
%}
%\bigskip

%\bigskip
%\cventry{2014}{Copy Real Url}{}{A browser extension}{}{
%\begin{description}
    %\item[$\bullet$] Google Search results and links(sent/received) in Facebook Chat are redirection urls. This extension is used to copy the actual target urls instead of the redirection urls.
%\item[$\bullet$] Implemented for Google Chrome and Mozilla Firefox. 
%\item[$\bullet$] Github: \httplink{github.com/ma08/chrome-link-copy}, \httplink{github.com/ma08/firefox-link-copy}. 
%\end{description}
%}



%\cventry{2014}{CASE(Computer Aided Software Engineering) Tool}{Software Engineering Project}{}{}{
%\begin{description}
%\item[$\bullet$] Developed a CASE Tool, a tool used to create(draw) data flow diagrams, modules, hierarchies etc using Java as a part of Software Engineering Course.
%\item[$\bullet$] Using this Tool user can draw Data Flow Diagrams and State Charts by drag and drop. The software automatically creates and updates Data Dictionary and also checks for Balancing Errors. 
%%\item[$\bullet$] Also allows users to Save diagrams, Edit diagrams, Print diagrams. 
%\end{description}
%}

%\bigskip
%\cventry{2013}{CASE(Computer Aided Software Engineering) Tool}{}{Software Engineering Project}{}{
%\begin{description}
%\item[$\bullet$] Developed using Java for Software Engineering Course by a team of 2. A tool used to create(draw) data flow diagrams, modules, hierarchies etc.
%\item[$\bullet$] Primarily used the Swing library to create the GUI to create the diagrams. 
%\item[$\bullet$] Automatic generation of Data Dictionary and checking of Balancing Error. 
%\item[$\bullet$] Implemented save and load operations for the diagrams using Serialization. 
%\item[$\bullet$] Github : \httplink{github.com/ma08/CASE}. 
%\end{description}
%}

%\bigskip



\section{Publications}
%\cventry{July 2017}{On the practical implementation of impossible differential cryptanalysis on reduced-round AES}{}{}{}{
%\begin{description}
 %   \item[$\bullet$] First practical implementation of impossible differential attack on 5-round AES.
%\item[$\bullet$] Published in International Conference on Applications and Techniques in Information Security 2017.
%\end{description}
%}
\cvline{July 2017}{\textbf{Sourya Kakarla}, Srinath Mandava, Dhiman Saha, and Dipanwita Roy Chowdhury. "On the practical implementation of impossible differential cryptanalysis on reduced-round AES." In \textit{International Conference on Applications and Techniques in Information Security, pp. 58-72. Springer, Singapore, 2017. DOI: 10.1007/978-981-10-5421-1\_6}}

\cvline{Aug 2016}{Chakraborty, Abhijnan, Bhargavi Paranjape, \textbf{Sourya Kakarla}, and Niloy Ganguly. "Stop clickbait: Detecting and preventing clickbaits in online news media." In \textit{2016 IEEE/ACM International Conference on Advances in Social Networks Analysis and Mining (ASONAM), pp. 9-16. IEEE, 2016. DOI: 10.1109/ASONAM.2016.7752207}}

\cvline{Sep 2020}{Aahan Dhabolkar, \textbf{Sourya Kakarla}, Dhiman Saha. "Looney Tunes: Exposing the Lack of DRM Protectionin Indian Music Streaming Services" Submitted to \textit{EuroSys 2021}}

\cvline{Dec 2018}{Saha, Dhiman, \textbf{Sourya Kakarla}, and Dipanwita Roy Chowdhury. "Dinamite: internal differential match-in-the-end attack on eight-round PAEQ." \textit{IET Information Security 13, no. 4 (2018): 378-388.  DOI: 10.1049/iet-ifs.2018.5033 }}

\cvline{Sep 2017}{Saha, Dhiman, \textbf{Sourya Kakarla}, Srinath Mandava, and Dipanwita Roy Chowdhury. "Gain: practical key-recovery attacks on round-reduced PAEQ." \textit{Journal of Hardware and Systems Security 1, no. 3 (2017): 282-296. DOI: 10.1007/s41635-017-0010-5}}

\cvline{Dec 2016}{Saha, Dhiman, \textbf{Sourya Kakarla}, Srinath Mandava, and Dipanwita Roy Chowdhury. "Gain: practical key-recovery attacks on round-reduced PAEQ." In \textit{International Conference on Security, Privacy, and Applied Cryptography Engineering, pp. 194-210. Springer, Cham, 2016. DOI: 10.1007/978-3-319-49445-6\_11}}



%\cventry{Aug 2016}{Stop clickbait: Detecting and preventing clickbaits in online news media}{}{}{}{
%\begin{description}
    %\item[$\bullet$] Developed the %\href{https://chrome.google.com/webstore/detail/stop-clickbait/iffolfpdcmehbghbamkgobjjdeejinma}{chrome extension} to identify clickbait articles linked across different websites and set up the server using Flask and nginx on AWS.
%\item[$\bullet$] Won the best student paper award at the conference IEEE/ACM ASONAM 2016.
%\end{description}
%}



\section{Technical Experience}
\subsection{Have Experience With}
\cvline{languages}{C, C++, C\#, Python, Java, JavaScript, Bash}
\cvline{technologies}{.NET, node.js, Flask, Angular, mongoDB, Microsoft Azure Service Fabric, SQL, OpenMP, \LaTeX, GNU Assembler, Verilog, Flex, Docker, Android Studio, i3 Window Manager}
%\section{Relevant Coursework}

%\cvline{Third Sem}{Algorithms-1, Discrete Structures, Introduction to Electronics, Signals and Systems}
%\cvline{Fourth Sem}{Switching Circuits, Probability and Statistics, Formal Languages and Automata Theory}
%\cvline{Fifth Sem}{Compilers, Computer Organization and Architecture, Algorithms-2}
%\cvline{Sixth Sem}{Operating Systems, Computer Networks, Database Management Systems}
%\cvline{Seventh Sem}{Natural Language Processing, Social Computing, Theory of Computation, Machine Learning, Artificial Intelligence}
%\cvline{Eighth Sem}{Foundations of Cryptography, High Performance Computer Architecture, Computational Geometry}
%\cvline{Ninth Sem}{Cryptography and Network Security, Parallel and Distributed Algorithms, Selected Topics in Algorithms}



\section{Activities}
\cventry{2013--2017}{Member, Football Team}{M.S.Hall}{IIT Kharagpur}{}{
\item[$\bullet$] Member of Gold Winning Team, M.S.Hall in the Sports General Championship '14.
}
\cventry{2013--2017}{Member, Opensoft Team}{M.S.Hall}{IIT Kharagpur}{}{
\begin{description}
 	\item[$\bullet$]   Captained the team in the academic year 2015-2016.
 	\item[$\bullet$]   Member of Bronze Winning Team, M.S.Hall in the Tech General Championship '14.
 	\item[$\bullet$]   Member of Bronze Winning Team, M.S.Hall in the Tech General Championship '17.
\end{description}
}
\cventry{2012--2013}{Member, KGPian Game Theory Society}{IIT Kharagpur}{}{}{
\begin{description}
%\item[$\bullet$] Studied various aspects of Game Theory and its applications.
	\item[$\bullet$] Published an article on  Burning Bridges in the Football Transfer Market in  The Strategist, the magazine of the society.
\end{description}
}
\cventry{2018--}{Author, Poetry Blog }{\url{www.instagram.com/sourya4}}{}{}{}
\cventry{2019--}{Memeber, Chedis To Medley }{An amateur music band based out of IIT KGP insti}{}{}{}
\cventry{2020--}{Completed Inner Engineering offered by Sadhguru, Isha Foundation}{\url{https://www.innerengineering.com/}}{}{}{}

%\section{References}
%\cventry{}{Prof. Dipanwita Roy Chowdhury}{CSE Dept, IIT Kharagpur}{}{}{
%Email: drc@cse.iitkgp.ernet.in
%Phone: 9434016982
%}
%Studied various aspects of Game Theory and its applications.
%\cvline{}{, CSE Dept. IIT Kharagpur.  }
%\cventry{2008--2010}{Football Team}{V.P.S Public School}{}{}{
%Part of school team in 9th and 10th class. Took part in an Interschool Tournament and reached the semifinals.
%}

%\cventry{}{Other Current Hobbies}{}{}{}{
%Web Programming, Camping, Hiking, Backpacking, Traveling, Road Trips, Basketball, Piano, Guitar
%}

\end{document}
